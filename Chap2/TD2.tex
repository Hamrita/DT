\documentclass[12pt]{article}
\usepackage[french]{babel}
\usepackage[T1]{fontenc}
\usepackage[utf8]{inputenc}
\usepackage{amsmath,amssymb}
\usepackage[top=1.5cm, left=2cm, right=1.5cm, bottom=1.75cm]{geometry}
\usepackage{booktabs}
\usepackage{array}
\usepackage{enumitem}



\title{Série d'exercices des travaux dirigés 2\\ Outils d'aide à la décision}
\author{IHEC Sousse - L3: Management}
\date{2026}

\begin{document}

\maketitle

\section*{Objectifs pédagogiques}
\begin{itemize}
\item Maîtriser la formulation de problèmes réels en programmation linéaire
\item Transformer entre formes canonique et standard
\item Résoudre graphiquement des PL à deux variables
\item Interpréter économiquement les solutions optimales
\item Reconnaître les différents types de contraintes en économie
\end{itemize}

\section*{Exercice 1 : Optimisation de production}

Une usine fabrique deux types de composants électroniques : A et B. Chaque composant nécessite du temps de fabrication sur deux machines différentes :
\begin{itemize}
\item Composant A : 2 heures sur Machine 1 et 1 heure sur Machine 2
\item Composant B : 1 heure sur Machine 1 et 3 heures sur Machine 2
\end{itemize}
Les disponibilités sont : 
\begin{itemize}
\item Machine 1 : 100 heures/semaine
\item Machine 2 : 90 heures/semaine
\end{itemize}

Les profits unitaires sont : 30€ pour A et 50€ pour B.

\begin{enumerate}
\item Formuler le programme linéaire qui maximise le profit.
\item Mettre sous forme canonique.
\item Mettre sous forme standard avec variables d'écart.
\item Représenter graphiquement la région réalisable.
\item Déterminer la solution optimale graphiquement.
\end{enumerate}

\section*{Exercice 2 : Planification de régime alimentaire}

Une clinique souhaite minimiser le coût d'un régime alimentaire tout en respectant des contraintes nutritionnelles. Deux aliments sont disponibles :
\begin{itemize}
\item Aliment X : 2€/kg, contient 3 unités de protéines et 1 unité de vitamines par kg
\item Aliment Y : 3€/kg, contient 1 unité de protéines et 2 unités de vitamines par kg
\end{itemize}

Les besoins minimums sont : 
\begin{itemize}
\item 9 unités de protéines par jour
\item 8 unités de vitamines par jour
\end{itemize}

\begin{enumerate}
\item Formuler le programme linéaire de minimisation.
\item Transformer en problème de maximisation équivalent.
\item Représenter la région réalisable.
\item Trouver la solution optimale graphiquement.
\item Calculer le coût minimum quotidien.
\end{enumerate}

\section*{Exercice 3 : Allocation de budget marketing}

Une entreprise dispose d'un budget de 50 000€ pour la publicité sur deux supports :
\begin{itemize}
\item Radio : coût 1000€ par spot, touche 10 000 personnes
\item TV : coût 5000€ par spot, touche 50 000 personnes
\end{itemize}

Contraintes : 
\begin{itemize}
\item Maximum 30 spots radio
\item Maximum 10 spots TV
\item Au moins 300 000 personnes doivent être touchées
\end{itemize}

L'objectif est de minimiser le coût total.

\begin{enumerate}
\item Définir les variables de décision.
\item Formuler le programme linéaire.
\item Mettre sous forme canonique.
\item Représenter graphiquement.
\item Déterminer la stratégie optimale.
\end{enumerate}

\section*{Exercice 4 : Problème de transport}

Une entreprise a deux usines (U1 et U2) et trois centres de distribution (C1, C2, C3).
\begin{itemize}
\item Capacités de production : U1 = 100 unités, U2 = 150 unités
\item Demandes : C1 = 80 unités, C2 = 120 unités, C3 = 50 unités
\item Coûts de transport (€/unité) :
\end{itemize}
\begin{center}
\begin{tabular}{c|ccc}
\toprule
& C1 & C2 & C3 \\
\midrule
U1 & 4 & 6 & 8 \\
U2 & 6 & 4 & 7 \\
\bottomrule
\end{tabular}
\end{center}
\begin{enumerate}
\item Formuler le programme linéaire pour minimiser les coûts de transport.
\item Identifier le nombre de variables et de contraintes.
\item Expliquer pourquoi ce problème nécessite des contraintes d'égalité.
\item Mettre sous forme standard.
\item Analyser graphiquement un sous-problème simplifié à 2 usines et 2 centres.
\end{enumerate}
\section*{Exercice 5 : Gestion de portefeuille d'investissement}
Un investisseur dispose de 100 000€ à répartir entre trois types d'actifs :
\begin{itemize}
\item Actions A : rendement 8\%, risque élevé (coefficient 3)
\item Obligations B : rendement 5\%, risque moyen (coefficient 2)
\item Placements C : rendement 3\%, risque faible (coefficient 1)
\end{itemize}

Contraintes :
\begin{itemize}
\item Maximum 40\% en actions A
\item Minimum 20\% en placements C
\item Le risque total moyen ne doit pas dépasser 2 (calculé comme moyenne pondérée des coefficients)
\end{itemize}

Objectif : Maximiser le rendement total.

\begin{enumerate}
\item Formuler le programme linéaire.
\item Expliquer l'hypothèse de divisibilité dans ce contexte.
\item Mettre sous forme canonique.
\item Représenter graphiquement pour deux variables (en fixant la troisième).
\item Interpréter économiquement la solution optimale.
\end{enumerate}


\end{document}