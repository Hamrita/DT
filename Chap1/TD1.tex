\documentclass[a4paper,12pt]{article}
\usepackage[french]{babel}
\usepackage[T1]{fontenc}
\usepackage[utf8]{inputenc}
\usepackage{amsmath,amssymb}
\usepackage{booktabs}
\usepackage[top=0.5cm, bottom=1.5cm, left=1.75cm, right=1cm]{geometry}
%\geometry{margin=2cm}

\title{Série d'exercices de Travaux Dirigés 1 \\ Outils d'aide à la décision }
\author{ IHEC Sousse}
\date{2026}

\begin{document}

\maketitle

\section*{Objectifs pédagogiques}
\begin{itemize}
\item Maîtriser la modélisation d’un problème de décision.
\item Appliquer le concept de dominance pour réduire l’ensemble des choix.
\end{itemize}

\section*{Exercice 1 : Formulation d’un problème de décision}
Un agriculteur doit choisir entre trois types de cultures (blé, maïs, tournesol) pour la saison à venir. Les profits dépendent des conditions climatiques (sècheresse, normale, pluvieuse).

\[
A = \{a_1: \text{blé}, a_2: \text{maïs}, a_3: \text{tournesol}\}
\]
\[
E = \{e_1: \text{sècheresse}, e_2: \text{normale}, e_3: \text{pluvieuse}\}
\]

Les profits (en milliers d’euros) sont donnés par la matrice suivante :

\begin{center}
\begin{tabular}{c|ccc}
\toprule
Action / État & $e_1$ & $e_2$ & $e_3$ \\
\midrule
$a_1$ & 20 & 60 & 40 \\
$a_2$ & 30 & 50 & 70 \\
$a_3$ & 10 & 80 & 30 \\
\bottomrule
\end{tabular}
\end{center}

\textbf{Questions :}
\begin{enumerate}
\item Écrire l’ensemble $A$ et $E$.
\item Calculer $c(a_2, e_3)$.
\end{enumerate}

\vspace{1cm}

\section*{Exercice 2 : Dominance entre actions}
Soit la matrice de décision suivante :

\begin{center}
\begin{tabular}{c|ccc}
\toprule
Action / État & $e_1$ & $e_2$ & $e_3$ \\
\midrule
$a_1$ & 50 & 30 & 40 \\
$a_2$ & 60 & 20 & 50 \\
$a_3$ & 40 & 40 & 30 \\
$a_4$ & 55 & 25 & 45 \\
\bottomrule
\end{tabular}
\end{center}

\textbf{Questions :}
\begin{enumerate}
\item Vérifier si $a_2$ domine $a_1$.
\item Vérifier si $a_4$ domine $a_2$.
\item Déterminer l’ensemble des actions efficaces $A^*$.
\end{enumerate}

\newpage
\vspace*{1cm}

\section*{Exercice 3 : Application de la dominance}
Un investisseur hésite entre quatre projets d’investissement (P1, P2, P3, P4). Les gains dépendent de l’évolution du marché (hausse, stabilité, baisse).

Matrice des gains (en k€) :

\begin{center}
\begin{tabular}{c|ccc}
\toprule
Projet / Marché & Hausse & Stabilité & Baisse \\
\midrule
P1 & 100 & 60 & 20 \\
P2 & 80 & 70 & 40 \\
P3 & 90 & 50 & 30 \\
P4 & 70 & 80 & 50 \\
\bottomrule
\end{tabular}
\end{center}

\textbf{Questions :}
\begin{enumerate}
\item Y a-t-il une action qui domine strictement une autre ? Justifier.
\item Déterminer $A^*$.
\item Si on retire P4, l’ensemble des actions efficaces change-t-il ?
\end{enumerate}


\end{document}